\documentclass[paper=a4, fontsize=11pt, parskip=full]{scrartcl} % A4 paper and 11pt font size

\usepackage[T1]{fontenc} % Use 8-bit encoding that has 256 glyphs
%\usepackage{fourier} % Use the Adobe Utopia font for the document - comment this line to return to the LaTeX default
\usepackage[english]{babel} % English language/hyphenation
\usepackage{amsmath,amsfonts,amsthm} % Math packages

\usepackage{graphicx}

\usepackage{float}

%Preamble
\usepackage{listings}
\usepackage{color}
\definecolor{javared}{rgb}{0.6,0,0} % for strings
\definecolor{javagreen}{rgb}{0.25,0.5,0.35} % comments
\definecolor{javapurple}{rgb}{0.5,0,0.35} % keywords
\definecolor{javadocblue}{rgb}{0.25,0.35,0.75} % javadoc
 
\lstset{language=Java,
basicstyle=\ttfamily,
keywordstyle=\color{javapurple}\bfseries,
stringstyle=\color{javared},
commentstyle=\color{javagreen},
morecomment=[s][\color{javadocblue}]{/**}{*/},
numbers=left,
numberstyle=\tiny\color{black},
stepnumber=2,
numbersep=10pt,
tabsize=4,
showspaces=false,
showstringspaces=false}


\usepackage{hyperref}
\hypersetup{
    colorlinks=true,
    linkcolor=blue,
    filecolor=magenta,      
    urlcolor=cyan,
}

\usepackage{lipsum} % Used for inserting dummy 'Lorem ipsum' text into the template

\usepackage{sectsty} % Allows customizing section commands
\allsectionsfont{\centering \normalfont\scshape} % Make all sections centered, the default font and small caps

\usepackage{fancyhdr} % Custom headers and footers
\pagestyle{fancyplain} % Makes all pages in the document conform to the custom headers and footers
\fancyhead{} % No page header - if you want one, create it in the same way as the footers below
\fancyfoot[L]{} % Empty left footer
\fancyfoot[C]{} % Empty center footer
\fancyfoot[R]{\thepage} % Page numbering for right footer
\renewcommand{\headrulewidth}{0pt} % Remove header underlines
\renewcommand{\footrulewidth}{0pt} % Remove footer underlines
\setlength{\headheight}{13.6pt} % Customize the height of the header

\numberwithin{equation}{section} % Number equations within sections (i.e. 1.1, 1.2, 2.1, 2.2 instead of 1, 2, 3, 4)
\numberwithin{figure}{section} % Number figures within sections (i.e. 1.1, 1.2, 2.1, 2.2 instead of 1, 2, 3, 4)
\numberwithin{table}{section} % Number tables within sections (i.e. 1.1, 1.2, 2.1, 2.2 instead of 1, 2, 3, 4)

\setlength\parindent{0pt} % Removes all indentation from paragraphs - comment this line for an assignment with lots of text

%----------------------------------------------------------------------------------------
%	TITLE SECTION
%----------------------------------------------------------------------------------------

\newcommand{\horrule}[1]{\rule{\linewidth}{#1}} % Create horizontal rule command with 1 argument of height

\title{	
\normalfont \normalsize 
\textsc{University of Virginia, Department of Computer Science} \\ [25pt] % Your university, school and/or department name(s)
\horrule{0.5pt} \\[0.4cm] % Thin top horizontal rule
\huge Java Cheat Sheet \\ % The assignment title
\horrule{2pt} \\[0.5cm] % Thick bottom horizontal rule
}

\author{Dr. Mark R. Floryan} % Your name

\date{\normalsize\today} % Today's date or a custom date

\begin{document}

\maketitle % Print the title


\section{Java: General Things to Remember}

\begin{enumerate}
	\item All code must be inside of a class definition (except import and package statements).
	\item The name of the class in a file must match the name of the file. For example, "public class LinkedList" must be in a file called "LinkedList.java"
	\item Classes can contain a method "public static void main(String[] args)" as an entry point to the whole program.
	\item Whitespace does NOT matter in java. The compiler will completely ignore all whitespace.
\end{enumerate}
%------------------------------------------------

\section{Primitive Data Types}

The primitive variable types are:

\begin{lstlisting}
int x = 5; //integers
double d = 3.4; //decimal values
char c = 'h'; //characters. Use single quotes.
boolean b = false; //true or false

/* Other much less commonly used */
byte b = 24b;
short s = -8s;
long l = 2000000;
float = 4.567;
\end{lstlisting}

Some examples of using these data types include:

\begin{lstlisting}
int x; 		//automatically set to 0 by default
x++; 		//increment integer by 1
x--; 		//decrement integer by 1

int z = 14;

int total = (x + z) * x; 	//expressions
int remainder = x % z; 		//remainder after x / z

/* Boolean operators */
boolean b1 = false;
boolean b2 = true;
boolean result;

result = b1 && b2; 	//logical AND
result = b1 || b2; 	//logical OR
result = !b1; 		//logical negation
\end{lstlisting}



\section{Input Output}

Input from the keyboard can be done like this:

\begin{lstlisting}
Scanner in = new Scanner(System.in); 	//make a scanner object
int x = in.nextInt();					//read int from keyboard
double y = in.nextDouble();				//read double from keyboard
float f = in.nextFloat();				//read float from keyboard
boolean b = in.nextBoolean();			//read bool from keyboard
long l = in.nextLong();					//read long from keyboard
String s = in.next();					//read string from keyboard
\end{lstlisting}

Input from a file can be done like this:

\begin{lstlisting}
BufferedReader in =
				new BufferedReader(new FileReader("inputfile.txt")); 
String text = in.readLine(); 	//reads the next line
in.close();
\end{lstlisting}

Output to the console is done like this:

\begin{lstlisting}
//prints text concatenated with x
System.out.print("The answer is " + x);

//prints and moves cursor to next line
System.out.println("something else");
\end{lstlisting}

Output to a file can be done like this:

\begin{lstlisting}
PrintWriter outFile =
				new PrintWriter(new FileWriter("outputfile.txt"))); 
outFile.print("Hello "); 
outFile.println("world"); 
outFile.close();
\end{lstlisting}


\section{Strings}

Strings are reference types in Java (so they are NOT primitives).

\begin{lstlisting}
String s1 = "Hello";			//example string
String s2;						//"" empty string by default
String s3 = new String("Hi");	//Also makes a string
\end{lstlisting}

Common operators on strings include:

\begin{lstlisting}
String result;

result = s1 + s2; 	//"hi " + "there" = "hi there"
s1.length();		//returns length of string
s1.charAt(2);		//accesses the char at position 2 (indexed from 0)
s1.substring(1,3);	//part of string starting at index 1, length 3
s1.equals(s2);		//compare strings using this structure
s1.toUpperCase();	//returns the string as all uppercase
s1.toLowerCase();	//returns the string as all lowercase
\end{lstlisting}

\section{Converting Between Data Types}

In Java, we often need to convert between different types of variables. Here are some common conversions:

\begin{lstlisting}
/* int (or any other primitive) to string */
int x = 5;
String s = "" + x;	//"" + variable concatenates as a string

/* String version of number to int or double */
int i = Integer.parseInt("123");		//converts string "123" into integer 123
double d = Double.parseDouble("3.14")	//converts string "3.14" into double 3.14

/* dividing integers does integer division */
int x1 = 3;
int x2 = 5;
double result = x1 / x2; 			// 3/5=(int)0.6 = 0
result = (double)x1 / (double)x2;	//0.6

/* double to int */
double x = 3.467;
int y = (int)x;		//y is 3, decimal truncated
\end{lstlisting}


\section{Arrays}

Three primary ways to instantiate arrays:

\begin{lstlisting}
double[] dArray = new double[5];	//know the size, but not contents
int[] oddNumbers = {1,3,5,7,9};		//know the contents already

int x;
/* Stuff here */
String[] sArray = new String[x];	//use variable to intialize array

int[][] = new int[5][4];			//two-dimensional array
\end{lstlisting}

Some common things we do with arrays include:

\begin{lstlisting}
x[3] = 5;	//Access array at position 3, set to 5.
x.length;	//get the number of elements in array

/* How to loop through an array */
for(int i=0; i<x.length; i++){
	System.out.println(x[i]);
}
\end{lstlisting}

\section{Java Math Library}

Java contains several functions inside the Math class that are useful. Among them:

\begin{lstlisting}
Math.abs(x);			//absolute value of x
Math.max(a,b);			//larger of a and b
Math.min(a,b);			//smaller of a and b
Math.sin(theta);		//sin trig function
Math.cos(theta);		//cos trig function
Math.tan(theta);		//tangent trig function
Math.toRadians(deg);	//convert deg to radians
Math.toDegrees(rad);	//convert rad to degrees
Math.exp(x);			//raises e^x
Math.log(x);			//natural logarithm
Math.pow(a,b);			//raise a to power of b
Math.sqrt(a);			//square root of a
Math.E;					//value of constant e
Math.PI;				//pi
\end{lstlisting}



%------------------------------------------------

%----------------------------------------------------------------------------------------

\end{document}


%----------------------------------------------------------------------------------------
%----------------------------------------------------------------------------------------
%----------------------------------------------------------------------------------------
%----------------------------------------------------------------------------------------
%----------------------------------------------------------------------------------------
%----------------------------------------------------------------------------------------


%WORKS CITED:

%%%%%%%%%%%%%%%%%%%%%%%%%%%%%%%%%%%%%%%%%
% Short Sectioned Assignment
% LaTeX Template
% Version 1.0 (5/5/12)
%
% This template has been downloaded from:
% http://www.LaTeXTemplates.com
%
% Original author:
% Frits Wenneker (http://www.howtotex.com)
%
% License:
% CC BY-NC-SA 3.0 (http://creativecommons.org/licenses/by-nc-sa/3.0/)
%
%%%%%%%%%%%%%%%%%%%%%%%%%%%%%%%%%%%%%%%%%

%----------------------------------------------------------------------------------------
%	PACKAGES AND OTHER DOCUMENT CONFIGURATIONS
%----------------------------------------------------------------------------------------